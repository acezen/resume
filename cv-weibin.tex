% !TEX TS-program = xelatex
% !TEX encoding = UTF-8 Unicode
% !Mode:: "TeX:UTF-8"

\documentclass{resume}
\usepackage{zh_CN-Adobefonts_external} % Simplified Chinese Support using external fonts (./fonts/zh_CN-Adobe/)
%\usepackage{zh_CN-Adobefonts_internal} % Simplified Chinese Support using system fonts
\usepackage{linespacing_fix} % disable extra space before next section
\usepackage{cite}

\begin{document}
	\pagenumbering{gobble} % suppress displaying page number
	
	\name{曾维彬}
	
	\basicInfo{
		\email{weibin.zen@gmail.com} \textperiodcentered\ 
		\phone{(+86) 130-5192-0810} \textperiodcentered\ 
		\linkedin[Weibin Zeng]{https://www.linkedin.com/in/weibin-zeng-425aa9141}}
	
	\section{\faGraduationCap\  教育背景}
	\datedsubsection{\textbf{北京航空航天大学},\ 北京}{2017 -- 至今}
	\textit{在读硕士研究生}\ 计算机技术, 预计2020 年 1 月毕业
	\datedsubsection{\textbf{西安电子科技大学},\ 西安}{2011 -- 2015}
	\textit{学士}\ 电子信息科学与技术
	
	\section{\faUsers\ 项目经历}
	\datedsubsection{\textbf{GRAPE大规模图计算系统}}{2017年3月 -- 2019年3月}
	%\role{Golang, Linux}{个人项目,和富帅糕合作开发}
	\begin{onehalfspacing}
	搜索风控部门标签传播业务应用合作开发
		\begin{itemize}
			\item 百万量级别的user--item二部图数据,现在所使用的ODPS模型性能较低(1天的数据量计算一次需要近一分钟);
			\item 针对特定的二部图数据优化了载图模块,增加了多线程并行载图实现;集成Kafka,实现数据的实时更新;针对LPA算法编写了相应的PIE模型;
			\item 1天数据量,GRAPE计算时间为\~{}1s, 相对ODPS模型的\~{}50s,性能提升了近50倍;30天数据量,ODPS模型为1h+, GRAPE可达到\~{}1min;
		\end{itemize}
	\end{onehalfspacing}

	\begin{onehalfspacing}
	拉卡拉信用卡数据挖掘应用开发
		\begin{itemize}
			\item 信用卡数据为多类型顶点数据,给定一个路径模式,挖掘出所有符合该模式的数据;
			\item 在原单类型顶点子图结构上增加多类型顶点的支持;依据需求设计并实现路径挖掘算法,增加了批量处理多路径模式挖掘的功能;
			\item TB级数据,GRAPE的处理时间为\~{}5min, 相对于拉卡拉现有算法(1h+),性能提升了近十倍,并且支持批量处理能力;
		\end{itemize}
	频繁子图发掘算法实现
		\begin{itemize}
			\item GRAPE系统缺乏子图挖掘的应用范例,gSpan算法是频繁子图挖掘领域的经典算法;
			\item 依据gSpan算法论文,在PIE模型实现并行gSpan算法,并在并行通信上进行优化;
			\item 在分布式并行运算下,GRAPE gSpan算法性能上相对已实现的单机算法有着线性倍数的提升;
		\end{itemize}
	\end{onehalfspacing}
	
	\section{\faUsers\ 工作实习经历}
	\datedsubsection{\textbf{阿里巴巴达摩院智能计算实验室},\ 北京}{2019年1月 -- 2019年3月}
	\role{实习开发工程师}{搜索风控部门标签传播业务应用合作开发}
	
	\datedsubsection{\textbf{七桥科技},\ 北京}{2017年3月 -- 2018年10月}
	\role{实习开发工程师}{GRAPE大规模图计算系统协助开发}
	
	\datedsubsection{\textbf{金证科技},\ 深圳}{2015年7月 -- 2016年9月}
	\role{C++助理开发工程师}{U版证券集中交易系统的协助开发与维护,深港通业务模块的协作开发}

	% Reference Test
	%\datedsubsection{\textbf{Paper Title\cite{zaharia2012resilient}}}{May. 2015}
	%An xxx optimized for xxx\cite{verma2015large}
	%\begin{itemize}
	%  \item main contribution
	%\end{itemize}
	
	\section{\faCogs\ IT 技能}
	% increase linespacing [parsep=0.5ex]
	\begin{itemize}[parsep=0.5ex]
		\item 编程语言:\ C/C++\ >\ Python
		\item 平台: Linux
		\item 开发: 大规模分布式图计算系统
	\end{itemize}
	
	%\section{\faHeartO\ 获奖情况}
	%\datedline{\textit{第一名}, xxx 比赛}{2013 年6 月}
	%\datedline{其他奖项}{2015}
	
	%\section{\faInfo\ 其他}
	% increase linespacing [parsep=0.5ex]
	%\begin{itemize}[parsep=0.5ex]
	%	\item 技术博客: http://blog.yours.me
	%	\item GitHub: https://github.com/username
	%	\item 语言: 英语 - 熟练(TOEFL xxx)
	%\end{itemize}
	
	%% Reference
	%\newpage
	%\bibliographystyle{IEEETran}
	%\bibliography{mycite}
\end{document}
